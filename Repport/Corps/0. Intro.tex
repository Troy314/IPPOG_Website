\chapter*{Introduction}\label{part:intro}
\addcontentsline{toc}{chapter}{\nameref{part:intro}}

The “Resource portal” was created in 2025, at the request of the HEP community during the ICHEP 2024 conference\footnote{
Education and Outreach Session @ICHEP 2024: \href{https://indico.cern.ch/event/1291157/contributions/5958376/}{https://indico.cern.ch/event/1291157/contributions/5958376/}}. It aims both to give an idea of outreach projects for the scientific community and to give a fair impression of the engagement and effort of the HEP outreach community.

The strategy for this resource portal is to serve as an online project database in which each outreach project is described in a post. We decided to start by uploading the projects presented by their owners and developers during IPPOG meetings' "success stories" sections. The idea here is to direct to a presentation by the authors themselves, possibly supplemented by a few resources\footnote{Presentation by Claire Adam @ICRC25: \href{https://indico.cern.ch/event/1258933/contributions/6475917/attachments/3105903/5509177/IPPOG-ICRC-ClaireAdam.pdf}{https://indico.cern.ch/event/1258933/contributions/6475917}}. We then enlarged it to include outreach parallel sessions of international conferences such as:
\begin{itemize}
    \item ICHEP – International Conference on High Energy Physics\footnote{ICHEP: \href{https://pos.sissa.it/cgi-bin/reader/family.cgi?code=ichep}{https://pos.sissa.it/cgi-bin/reader/family.cgi?code=ichep}}
    \item ICRC – International Cosmic Ray Conference\footnote{ICRC: \href{https://pos.sissa.it/cgi-bin/reader/family.cgi?code=icrc}{https://pos.sissa.it/cgi-bin/reader/family.cgi?code=icrc}}
    \item HEP-EPS – European Physical Society Conference on High Energy Physics\footnote{HEP-EPS: \href{https://pos.sissa.it/cgi-bin/reader/family.cgi?code=hep}{https://pos.sissa.it/cgi-bin/reader/family.cgi?code=hep}}
    \item LHCP – Large Hadron Collider Physics\footnote{LHCP: \href{https://pos.sissa.it/cgi-bin/reader/family.cgi?code=lhcp}{https://pos.sissa.it/cgi-bin/reader/family.cgi?code=lhcp}}
\end{itemize}

\bigskip

The website is built on the new CERN WordPress infrastructure\footnote{CERN WordPress infrastructure: \href{https://wordpress.docs.cern.ch/}{https://wordpress.docs.cern.ch/}}, which should replace the CERN Drupal websites from July 2025 onwards.

This document is divided into two parts. The first part (\ref{part:general}) gives a general explanation of the website and its current status (Chapter \ref{chap:context}). The second part (\ref{part:tech}) is a technical description of the website and of the upload process. Its first chapter gives an overview of the website's structure (Chapter \ref{chap:structure}), the second describes the whole automated process (Chapter \ref{chap:process}) from the database to uploading the projects on the website, and finally, the third chapter offers a visualization of the data uploaded on the website (Chapter \ref{chap:data}). The last chapter reviews useful links related to the website (Chapter \ref{chap:links}).